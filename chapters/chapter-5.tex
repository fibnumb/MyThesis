\chapter{Jet Results and Discussion} \label{ch:analysis}

Beginning in March of 2012, the LHC began a seven month run of pp collisions $\sqrt{s} = \,$ 8 TeV.  The jet cross sections and ratios of the cross sections for jets of different radii offers a unique perspective on the pQCD effects of hadronization at this new energy frontier.  Due to the expectation that no QGP is formed in a pp collision these measurements serve as a baseline for separating phenomena associated with the QGP in heavy-ion collisions.  In order to measure the jet cross section the following formula is used,

\begin{equation}
	\frac{d \sigma^{jet}}{d\eta \, dp_{T}} = \frac{A_{trigger}}{\epsilon_{trigger}(p_{T})} \times C_{MC} \times \frac{1}{A(p_{T}) } \times \frac{1}{L_{int}} \times \frac{dN^{jet}}{dp_{T} \, d\eta}
\label{eq:xsecdef}
\end{equation}

\noindent
where,

\begin{itemize}
  \item $A_{trigger}$ is the acceptance for EMCal triggered events and $\epsilon_{trigger}(p_{T})$ is the EMCal trigger efficiency.  These factors correct for imperfections in the electronics of the EMCal and the overall factors are equal to one in minnimum bias events.
  \item $C_{MC}$ is a correction factor due to detector effects and it allows for comparisons between the ALICE experiment to other experiments or theoretical calculations.  Unfolding is used to determine this factor.
  \item $L_{int}$ is the integrated luminosity during the period when the data was recorded.
  \item $A(p_{T})$ is the geometrical detector acceptance.
  \item $\frac{dN^{jet}}{dp_{T} \, d\eta}$ is the inclusive jet momentum spectra.
  
\end{itemize}

The following sections will go over how each factor was determined and the quality assurance procedures used for this analysis.

\section{Data Quality}
ALICE is a state-of-the-art experiment with excellent tracking and particle identification capabilities as discussed in Chapter \ref{ch:alice}.  However, just like any real world experiment, it contains a number of inefficiencies and imperfections.  This means that the data collected during the 8 TeV pp collison must be analyzed and cleaned in order to make hard physics conclusions.  

\section{Event Selection}

During the 8 TeV data collection period approximately 

\begin{table}[hb]
\caption{Summary of 8 TeV data taking period.}
\label{tab:table-a}
\begin{center}
\begin{tabular}[b]{|c|c|c|}
	\hline
	Period & \# of runs & \# of Min Bias events \\ \hline
	LHC12c & 89 & $\sim$24 M \\ \hline
	LHC12d & 140 & $\sim$62 M \\ \hline
	LHC12e & 5 & $\sim$2 M \\ \hline
	LHC12f & 56 & $\sim$15 M \\ \hline
	LHC12g & 8 & $\sim$0.4 M \\ \hline
	LHC12h & 159 & $\sim$75 M \\ \hline
	LHC12i & 40 & $\sim$3 M \\ \hline
	Total & 497 & $\sim$181 M \\ \hline

\end{tabular}
\end{center}
\end{table}

\section{Raw measurements}
The ALICE experiment is capable of two types of jet reconstruction, charged and full jets.  Charged jets use information from the charged particle tracking detectors, such as the ITS and TPC, in conjunction with a jet finding algorithm to identify jets.  Full jets implement a similar procedure but also incorporates the EMCal in order to 

\subsection{Raw Jet Momentum Spectra in pp Collisions}

\section{Unfolding}

\subsection{Corrections to particle Level}

\subsection{Unfolding Matrix}

\subsection{Unfolded Spectra}

\section{Trigger Efficiency}

\section{Systematic Uncertainties}

\subsection{Systematic Uncertainty to Jet Yield}

\subsection{Systematic Uncertainty to Jet Energy Scale}

\subsection{Total Uncertainty}

\section{Corrected pp jet cross section}

\subsection{Comparisons to pQCD predictions}

\subsection{Jet Cross Section and Ratios}
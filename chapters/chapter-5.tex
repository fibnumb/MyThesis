\chapter{Jet Results and Discussion} \label{ch:analysis}

Beginning in March of 2012, the LHC began seven months of pp collisions at $\sqrt{s} = \,$ 8 TeV.  The jet cross sections and ratios of the cross sections for jets of different radii offers a unique perspective on the pQCD effects of hadronization at this new energy frontier.  Due to the expectation that no QGP is formed in a pp collision these measurements serve as a baseline for separating phenomena associated with the QGP in heavy-ion collisions.  In order to measure the jet cross section the following formula is used,

\begin{equation}
	\frac{d^{2} \sigma^{jet}}{d\eta \, dp_{T}} = \frac{A_{trigger}}{\epsilon_{trigger}(p_{T})} \times C_{MC} \times \frac{1}{A(p_{T}) } \times \frac{1}{\mathscr{L}_{int}} \times \frac{dN^{jet}}{dp_{T} \, d\eta}
\label{eq:xsecdef}
\end{equation}

\noindent
where,

\begin{itemize}
  \item $A_{trigger}$ is the acceptance for EMCal triggered events and $\epsilon_{trigger}(p_{T})$ is the EMCal trigger efficiency.  These factors correct for imperfections in the electronics of the EMCal and the overall factors are equal to one in minimum bias events.
  \item $C_{MC}$ is a correction factor due to detector effects and it allows for comparisons between the ALICE experiment to other experiments or theoretical calculations.  Unfolding is used to determine this factor.
  \item $\mathscr{L}_{int}$ is the integrated luminosity during the period when the data was recorded.
  \item $A(p_{T})$ is the geometrical detector acceptance.
  \item $\frac{dN^{jet}}{dp_{T} \, d\eta}$ is the inclusive jet momentum spectra.
  
\end{itemize}

\noindent
Furthermore, it is useful to define the ratio of cross sections,

\begin{equation}
\mathscr{R}(p_{T};R_{1},R_{2}) = \frac{d^{2}\sigma(p_{T};R_{1})/d\eta \, dp_{T}}{d^{2}\sigma(p_{T};R_{2})/d\eta \, dp_{T}}
\label{eq:xsecratio}
\end{equation}

\noindent
where $\sigma(p_{T};R_{1})$ refers to the doubly differential cross section (Equation \ref{eq:xsecdef}) of a jet with radius $R_{1}$.  The ratio is carried out on a bin--by--bin basis per each $p_{T}$ bin.  

\section{Raw Jet Spectra}

This thesis reports inclusive jet results for radii between 0.1 and 0.5.  Furthermore, jet results for radii R = 0.2 and R = 0.4 will be presented in the body of this chapter while results from the other radii will be limited to the appendix.  Figure shows the raw (uncorrected) $p_{T}$ spectra for inclusive jets from both MB and EMCal triggered data.  It is also evident from Figure that the Emcla triggered data extends the $p_{T}$ reach 



\section{8 TeV Data Quality}
ALICE is a state-of-the-art experiment with excellent tracking and particle identification capabilities as discussed in Chapter \ref{ch:alice}.  However, just like any real world experiment, it contains a number of inefficiencies and imperfections.  This means that the data collected during the 8 TeV pp collision must be examined and any inaccuracies in the data must be removed before hard physics conclusions may be reached.  Data may be compromised at both the event-level, the experiment erroneously recorded something as an event, or at the constituent-level, one of the subdetectors mismeasured a feature of a particle, and these outliers must be accounted for and removed 

\section{Event Selection}

For an event to be selected into a physics analysis it must pass a number of quality control tests.  For example, the LHC must have be in a state of stable beams, cosmic rays must be excluded by only accepting tracks that originate from a vertex inside the detector, and the relevant detectors for a given analysis must be functioning as intended.  Event selction and QA is implemented via a centralized class, AliEventCuts, within the AliRoot framework.  This class contains a number of corrections including:

\begin{itemize}
  \item 
  \item Another entry in the list
\end{itemize}

Further more the class AliAnalysisTaskEmcalCorrection within the AliRoot framework performs a number of corrections to the tower-level within phyasics analysis dependent on the EMCal.  These corrections include:

\begin{itemize}
  \item Energy calibration:  Remove double counting due to the amount of energy typically deposited by a Hadron.
  \item Bad Channel Map: Remove both dead and hot towers from the EMCal clusterize based on an iterative average energy per occupancy algoritham
  \item Time Calibration:  Remove slow particles, such as neutrons, which readout to the EMCal after the event has been recorded.
   \item Exotics Correction:  Remove inefficiencies due to how a particle interacted with the EMCal.
\end{itemize}\

After the above cuts are performed on EMCal towers the cells compromising the towers are clustered together using a clusterization algoritham 



\begin{figure}[h]
\includegraphics[width=17cm]{8TeVRunefficency}
\centering
\caption{LHC state during the 8 TeV run. }
\label{fig:RunEff}
\end{figure}

During the 8 TeV data collection period approximately 180 million minimum bias events were recorded, as summarized in table \ref{tab:RunSummary}.  These events are separated into periods, which dictate the particular beam and detector configurations during the data taking.The 8 TeV data is broken into 7 periods with approximately 181 million minimum bias events recorded.  This minimum bias sample corresponds to an integrated luminosity, $\mathscr{L}_{int}$, of $8.95 \, pb^{-1}$ during this time period\cite{ALICE-PUBLIC-2017-002}.

\begin{table}[hb]
\label{tab:RunSummary}
\begin{center}
\begin{tabular}[b]{|c|c|c|}
	\hline
	Period & \# of runs & \# of Min Bias events \\ \hline
	LHC12c & 89 & $\sim \,$24 M \\ \hline
	LHC12d & 140 & $\sim \,$62 M \\ \hline
	LHC12e & 5 & $\sim \,$2 M \\ \hline
	LHC12f & 56 & $\sim \,$15 M \\ \hline
	LHC12g & 8 & $\sim \,$0.4 M \\ \hline
	LHC12h & 159 & $\sim \,$75 M \\ \hline
	LHC12i & 40 & $\sim \,$3 M \\ \hline
	Total & 497 & $\sim \,$181 M \\ \hline

\end{tabular}
\end{center}
\caption{2012 8 TeV data taking period.}
\end{table}

Approximately, 15\% of the data sampled is unusable due to malfunctions in TPC chambers, EMCal super modules, the electronics for the EMCal or TPC, and   
\section{Raw measurements}
The ALICE experiment is capable of two types of jet reconstruction, charged and full jets.  Charged jets use information from the charged particle tracking detectors, such as the ITS and TPC, in conjunction with a jet finding algorithm to identify jets.  Full jets implement a similar procedure but also incorporates the EMCal in order to 

\subsection{Raw Jet Momentum Spectra in pp Collisions}

\section{EMCal Triggered Data}

In addition with the minimum bias data collected, the EMCal was used during the 8 TeV run in order to provided an enhanced data set that is preferential to hard processes.   The Level-1 trigger\cite{Bourrion:2010js} in the EMCal has a associated trigger, $\epsilon$, of 

\begin{equation}
	\epsilon = \frac{N^{Triggered}_{events}}{N^{MinBias}_{events}} \times \frac{d^{2} N_{Triggered}^{jet}}{d\eta \, dp_{T}} \Bigg/  \frac{d^{2} N_{MinBias}^{jet}}{d\eta \, dp_{T}} 
\label{eq:xsecdef}
\end{equation}

\subsection{Acceptance Correction}
Jet spectra, cross sections, and ratios of cross sections are reported over the full azimuth angle and psuedorapidity acceptance.  However, due to jets being constrained to the EMCal, a geometric factor is used to correct for the limited acceptance of the detector.  This thesis uses a maximum jet radius of 0.5 to help study the effects of wide angle radiation on jet fragmentation.  Heavy-ion use smaller jet radii, typically of 0.2, to help negate the high multiplicity background.  Due to these geometric corrections the centroid of a jet is constrained to,

\begin{equation}
|\eta_{jet}| \leq 0.7 - R, \; 1.4 + R \leq \phi_{jet} \leq 3.14 -R.
\label{eq:jetconstration}
\end{equation}

\begin{equation}
A(p_{T}) = \frac{(1.4 - 2R) \times (1.745 - 2R)}{2 \pi}.
\label{eq:acceptance}
\end{equation}

For jets between R = 0.1 through R = 0.5 the following jet acceptance corrections are used.

\begin{table}[hb]
\label{tab:AcceptanceFactor}
\begin{center}
\begin{tabular}[b]{|c|c|c|}
	\hline
	Jet R & $A(p_{T})$ \\ \hline
	0.1 & 0.296 \\ \hline
	0.2 & 0.214\\ \hline
	0.3 & 0.146\\ \hline
	0.4 & 0.091\\ \hline
	0.5 & 0.048\\ \hline
\end{tabular}
\end{center}
\caption{EMCal jet acceptance for radii 0.1 - 0.5.}
\end{table}








\section{Unfolding}

The reconstructed jet $p_{T}$ has a number of detector effects `folded' into the measurement.  These effects included such things as:

\begin{itemize}
\item Tracking inefficiencies from the TPC and ITS.
\item Missing jet energy components from long-lived particles, such as the $K^{0}_{L}$ and neutron, that are cut by the EMCal timing requirement.
\item TPC track $p_{T}$ and EMCal cluster energy resolutions.
\item Hadronic corrections to the EMCal cluster spectrum.
\item Material loss in the detectors.
\end{itemize}

\noindent
Unfolding is the method by which these detector effects are removed from the raw inclusive jet spectra and a `true' jet spectra may be obtained and compared with theoretical calculations or other experimental results.  In order to unfold a jet spectra it is necessary to generate a response matrix that simulates the described effects above, after the response matrix is generated a number of statistical approaches including, Bayesian, Singular Value Decomposition (SVD), or Bin-by-Bin, may be applied to unfold the raw jet spectra.  In order to generate the response matrix we embed a Pythia generated event into a GEANT3 simulation of the ALICE detector.  Due to the fact that the performance and efficiency of the ALICE detector may change between the data taking periods each simulation is `anchored' to a given LHC, these anchors contain all the hot and dead sectors for the subdetectors, along with their calibrated performance during that specified data taking.  Two Monte Carlo data sets were produced with the MB trigger for the full 8 TeV run, the first was a Pythia generator using the Monesh-2013 tune and the second was a MB tune of the PHOjet Monte Carlo package.  Both data sets were explored for this thesis and it was decided that the final corrected spectra would be obtained via unfolding with the Pythia MC data set.  The magnitude of any one of the effects unfolding is supposed to account for is not expected to be very large, but combined may be significant, thus unfolding is an important step in this analysis.

\subsection{Response Matrix}
Given a truth-level particle jet $p_{T}$ we wish to reconstruct that jet's $p_{T}$ at the detector-level.  The particle-level pythia jets are constructed from the primary particles generated via Pythia while excluding any daughter decay particles in order to avoid double counting.  In addition the tracking efficiency in Pythia is known to deviate from nature.  This is due to Pythia under predicting the production of strange quarks.  
Constructing the response matrix in this case is calculated on a jet-by-jet basis.  The particle-level jet centroid ($\phi_{part}$,$\eta_{part}$)is matched to the detector-level jet via a constrain on the displaced distance between the two jet centroids in ($\phi$,$\eta$).  This distance was constrained to: $\Delta  R = \sqrt{(\phi_{part} - \phi_{det})^{2} + (\eta_{part} - \eta_{det})^{2}} \leq 0.25 \; $.  Once a jet is matched at the detector level to a jet generated from the particle level the response matrix is incremented by jet $p_{T}$ at both the detector and Monte Carlo levels.  The response matrix is generated with a fine binning with a width of 1 GeV per bin. 

\begin{figure*}[t!]
$\begin{array}{rl}
    \includegraphics[width=0.5\textwidth]{responseR02} &
    \includegraphics[width=0.5\textwidth]{responseR03}\\
    \multicolumn{2}{c}{\includegraphics[width=0.5\textwidth]{responseR04}}
\end{array}$
\caption[Response Matrices for R = 0.2, R=0.3, and R = 0.4 jets.]{\label{fig:response}Response Matrices for R = 0.2, R=0.3, and R = 0.4 jets.}
\end{figure*}

Figure \ref{fig:response} shows the response matrices for the R = 0.2 (top left), R = 0.3 (top right), and R = 0.4 (bottom) jets generated with the prescribed manner.  The response matrices display a linear relationship below 50 GeV on both axis and above ~ 100 GeV the matrices are statistics starved.  This is primarily due to the Monte Carlo Pythia and PhoJet data sets generated for the 8 TeV pp run did not model the high-$p_{T}$ triggers associated with the EMCal.  The particle jet finders configured for the response matrices allowed for jet finding down to a 100 MeV jet candidate at the particle level with no constraints on the minimum particle momentum or energy for a constituent.  The detector level jet finders were configured in the same manner as the jet finders configured for the raw jet spectra measurement.  

\subsection{Corrections to Particle Level}

Unfolding was performed using the \verb+RooUnfold+\cite{Adye:2011gm} software package.  Corrections are applied using the bin-by-bin\cite{Cowan:2002in} algorithm. 

\begin{equation}
C_{MC} \big( p_{T}^{low} : p_{T}^{high} \big) =  \frac{  \int^{p_{T}^{high}}_{p_{T}^{low}} dp_{T} \; \frac{dF^{uncorr}_{meas}}{dp_{T}} \times \frac{d^{2}N^{particle}_{MC}/d\eta \, dp_{T}}{d^{2}N^{detector}_{MC}/d\eta \, dp_{T}}  } { \int^{p_{T}^{high}}_{p_{T}^{low}} dp_{T} \; \frac{dF^{uncorr}_{meas}}{dp_{T}} }
\label{eq:binbybin}
\end{equation}

\noindent
where $d^{2}N^{particle}_{MC}/dp_{T} \, d\eta$ is the PYTHIA level inclusive jet spectra, $d^{2}N^{detector}_{MC}/dp_{T} \, d\eta$ is the GEANT 3 level inclusive jet spectra, $dF^{uncorr}_{meas} / dp_{T}$ is a weight function which minimizes the dependence on the two simulation spectra shapes, finally $p_{T}^{low}$ and $p_{T}^{low}$ are the lower and upper bin limits.  Due to the limited statistics derived from the Monte Carlos available the unfolding procedure was stable only in aunfolding the truth level jet spectra for the range: $p_{T,jet} \epsilon \;$ [10 GeV, 120 GeV] for both the raw Min Bias and Emcal triggered data sets.  Due to the  the final truth value for the jet spectra will be reported in this range.


\subsection{Unfolded MB Spectra}

\begin{figure}[h]
\includegraphics[width=17cm]{RawandUnfoldedSpectraMBR03}
\centering
\caption{Unfolded jet spectra with fine binning for R = 0.3}
\label{fig:Unfoldfine}
\end{figure}

Figure \ref{fig:Unfoldfine} shows an example of the output from the bin-by-bin unfodling with the fine binning for R = 0.3 jets.  It should be noted that at low-$p_{T}$ it was observed that unfolding increased the yield of the spectra while at high-$p_{T} \geq \,$ 40 GeV the yield was decreased for all jet radii in this analysis.  This is most likely due to the lack of statistics in the response matrix.


\subsection{Unfolded EMCal Triggered Spectra}


\section{Systematic Uncertainties}

Systematic uncertainties arise due to our limited knowledge of the precise operating conditions and performance of the experiment and also due to any bias in our understanding of how to fundamental model the interactions.  They systematics may therefore be broken into two components: uncertainties to the jet energy scale (JES) which shifts the momentum spectra along the x-axis and uncertainties in the jet yield which shift the spectra along the y-axis.  The systematical and statistical uncertainties presented in this analysis will be presented as errors to the yield of the spectra.  Due to the fact that the $p_{T}$ distribution follows a power law function, $dN/dp_{T} \sim p_{T}^{-5}$ uncertainties in the JES are converted to yield uncertainties by dividing each one by 5.
Due to the low statistics at the highest $p_{T}$ bins in this analysis, uncertainties in this regime my have large statistical fluctuations.  Small ,systematic variations for the input of the jet spectra will have a dramatic effect over sparsely filled bins versus bins with a low granularity.  As such it may be necessary to extrapolate the systematic from a low $p_{T}$ bin to those at the highsest $p_{T}$ range.  The systematics were performed on both the MB and EMCal triggered data samples but no large variation was observed between the two, thus only the uncertainties from the MB sample are shown and are extrapolated to the triggered data.


\subsection{Systematic Uncertainty to Jet Energy Scale}

\subsubsection{Tracking Efficiency}
Corrections for the tracking efficiency were performed by randomly throwing out 5\% of the tracks from each event from the 8 TeV data samples and reperforming jet on the altered data.  All of the inputs for jet finding were maintained.


\begin{figure*}[t!]
$\begin{array}{rl}
    \includegraphics[width=0.5\textwidth]{SysR02_TrkEff} &
    \includegraphics[width=0.5\textwidth]{SysR03_TrkEff}\\
    \multicolumn{2}{c}{\includegraphics[width=0.5\textwidth]{SysR04_TrkEff}}
\end{array}$
\caption[Systematic due to TPC tracking efficiency.]{\label{fig:trkeff}Systematic due to TPC tracking efficiency.}
\end{figure*}

\noindent
Figure \ref{fig:trkeff} shows the systematical uncertainties for R = 0.2 (top left), R = 0.3 (top right), and R = 0.4 (bottom) jets.   

\subsubsection{Hadronic Correction}

\subsubsection{EMCal Clusterization Algorithm}
\subsection{Systematic Uncertainty to Jet Yield}



\subsubsection{Luminosity Uncertainty}

The luminosity of a hadronic collider, $\mathscr{L}$, is given by the expression



\begin{equation}
\mathscr{L} = \frac{R}{\sigma}
\label{eq:xlumdef}
\end{equation}

\noindent
where R is the interaction rate and $\sigma$ is the visible cross section.  Due to the fact that we only measure events within a 10 cm window within the primary vertex region we must scale the total luminosity to that which is delievered within the primary vertex region of the ALICE experiment.  This scale factor is determined by dividing the total number of MB events to those accepted within the 10 cm window.  $N^{tot}_{MB} / N^{10 cm vertex}_{MB}$ = 1.024 from the acceptance criteria held in this analysis.
The luminosity along with its uncertainty were determined during a a special Van der Meer scan run in April of 2012\cite{ALICE-PUBLIC-2017-002}.  The total systematic uncertainty for the minimum bias (MB) trigger were obtained by measuring the visible cross section using the T0 and V0 detectors.  The MB trigger was defined as V0AND which required a hit in both tjhe V0A and V0C.  The cross section was reported as being a combined average for MB with the V0AND as, 

\begin{equation}
\sigma_{V0} = (55.8 \pm 1.2) mb
\label{eq:xlumdef}
\end{equation}

\noindent
with a combined systematic uncertainty of 2.19\% on the visible cross section and 2.60\% on the luminosity. 


\subsection{Total Uncertainty}

A summary of the total systematic errors used in the final analysis.

\begin{tabular}{ |p{5cm}||p{3cm}|p{3cm}|p{3cm}|  }
 \hline
 \multicolumn{4}{|c|}{Systematic Errors} \\
 \hline
 Systematic &R = 0.2 Jets & R = 0.3 Jets& R = 0.4 Jets\\
 \hline
Sensitivity to Clusterization   & AF    &AFG&   004\\
Hadronic Correction&   AX  & ALA   &248\\
Tracking Efficency &AL & ALB&  008\\
Sensitivity to Unfolding&DZ & DZA&  012\\
Momentum Resolution&   AS  & ASM&016\\
Energy Resolution& AND   &020 & 02\\
 Angola& AO  & AGO&024\\
 \hline

\end{tabular}
\noindent

The systematics from the yield and JES are added in quadrater together and this is combined in quardrater with the statistical errors.

\section{Corrected pp jet cross section}


\subsection{Comparisons to pQCD predictions}

\subsection{Jet Cross Sections and Ratios}




%Fucking left quote symbol ` and not '    

\chapter{Quantum Chromodynamics} \label{ch:qcd}
In 1968 deep inelastic scatterings performed at the Stanford Linear Accelerator Center showed that the proton had internal structure\cite{Riordan1287} called partons at the time.  Within a decade of this discovery the partons were broken into two categories: the mass carrying fermions were known as the quarks and the gauge boson force carriers were called gluons.  The interactions of these two types of particles were described by the quantum field theory known as Quantum Chromodynamics (QCD) and by the SU(3) symmetry group.  SU(3) guarantees that color charge is conserved and this results in quarks grouping together into `colorless' hadrons.

\section{The QCD Lagrangian}
QCD is the strongest of the known fundamental forces.  It is a gauge field theory described by the Lagrangian density

\begin{equation}
{\cal L}=-\frac{1}{4}F^{\alpha}_{\mu\nu}F^{\mu\nu}_{\alpha}
- \alpha_{s} (\bar{q}_{j}\gamma_{\mu}T_{\alpha}q_{j})G_{\alpha}^{\mu}
+ \bar{q}_{j}(i\gamma^{\mu} \partial_{\mu} - m)q_{j}
\label{eq:lagrangian}
\end{equation}

\noindent
where $q$ and $\bar{q}$ represent the color/anti-color fields summed over color $j$, $\alpha_{s}$ is the strong coupling strength,$\gamma^{\mu}$ is the Dirac gamma matrix, $G_{\alpha}^{\mu}$ is the gauge field for color \textit{$\alpha$}, $G_{\alpha}^{\mu}$ is similar in analogy to the \textbf{W} matrix from the electroweak theory.  $F^{\alpha}_{\mu\nu}$ is the field strength tensor and it describes the gluon interactions. The first term of the Lagrangian is the gluon contribution and carries no mass variable.  The second term describes how quarks and gluons interact with each other. The final term describes quark interactions and the coupling between them and will be explored further in this thesis.

At short distances, less than 0.2 \textit{fm}, the strong coupling constant becomes exceedingly small and the second term of the Lagrangian displays an important property known as asymptotic freedom\cite{Wilczek:2005az}.  Numerically the strong coupling constant is given as,

\begin{equation}
\alpha_{s} = \frac{1}{\beta_{0} \; \ln(Q^{2}/\Lambda^{2} )}
\label{eq:alpha_s}
\end{equation}

\noindent
where $\alpha_{s}$ is the strong coupling constant, $Q^{2}$ is the momentum transfer between two interacting partons, $\Lambda^{2}$ is a cutoff below which QCD phenomena are strongly suppressed, and $\beta_{0}$ is a scale factor.  Figure \ref{fig:as} shows the value of $\alpha_{s}$ as a function of the momentum transfer measured from various particle experiments and clearly shows the decreasing strength at high energies.

\begin{figure}[h]
\includegraphics[width=12.0cm]{alphas_s}
\centering
\caption{Strong coupling constant ($\alpha_{s}$) as a function of the momentum transfer (Q)\cite{CMS:2014mna}.}
\label{fig:as}
\end{figure}

\section{Jets}

Hard probes (large $Q^{2}$ interactions), are produced in the earliest stages of a high energy collision when the largest momentum transfer processes occur.  The interaction and scattering of partons is a 2 $\rightarrow$ 2 process, meaning that two partons will interact and the outgoing partons also come in pairs.  As two highly energetic partons propagate away from one another they will instigate a shower of daughter partons via gluon radiation and the generation of low-mass $q \bar{q}$\, pairs.  These daughter partons will go on to form collomated sprays of hadrons known as a `jet'.  If the jet was created in a high energy experiment, the final state hadrons will be recorded as tracks in a tracking detector or energy deposits in a calorimeter.  This process is shown in Figure \ref{fig:MakeAJet}.



\begin{figure}[h]
\includegraphics[width=12.0cm]{jetsatcmsand}
\centering
\caption{Diagram showing a jet created by two partons undergoing a hard scattering, forming into hadrons, and detected in a calorimeter\cite{JetPic}.}
\label{fig:MakeAJet}
\end{figure}

The physicist James Daniel Bjorken postulated that a correlation could be surmised by summing over the final state transverse momentum of the hadrons that form a jet to the parton that initiated the hard scattering\cite{PhysRev.179.1547}\cite{Bjorken:1973kd}.  This has led to jets becoming the work-horse for both experimentalists and theorists over the past 30 years in probing QCD phenomena.  This thesis makes use of jets as an important probe of QCD and the following sections are devoted to developing a background for both the theoretical and experimental treatment of jet physics.  The following sections of this chapter will be devoted to the background of jet production.

\subsubsection{Jet Production and The Factorization Theorem}\label{sec:fac}

Due to confinement bare quarks are unobserved, therefore experimentalists must probe QCD interactions by detecting the color neutral final state hadrons measured in collider experiments.  The factorization theorem allows for the final state jet cross section to be broken into a number of steps that can either be calculated pertubatively using pQCD or modeled phenomenologically.  Using the factorization theorem the jet cross section in a pp collision is:


\begin{equation}
d\sigma^{pp \rightarrow jet} \sim f_{a/A}(x_{1},Q^{2}) \otimes  f_{b/B}(x_{2},Q^{2}) \otimes d\sigma_{ab \rightarrow c + X} (x_{1},x_{2}) \otimes D_{c \rightarrow h/jet}(z,Q^{2})
\label{eq:xsection}
\end{equation}

\noindent
Breaking Equation \ref{eq:xsection} down we have:

\begin{itemize}
\item  $ f_{a/A}(x_{1},Q^{2})$ and $ f_{b/B}(x_{2},Q^{2})$ are the parton distribution functions (PDF) that describe the probability of finding parton, \textit{a} or \textit{b}, within nuclei, \textit{A} and \textit{B}, with a given momentum fraction, $x = p_{parton} / p_{hadron} $ as a function of $Q^{2}$.
\item  $d\sigma_{ab \rightarrow c + X} (x_{1},x_{2})$ is the pQCD parton-parton cross section due to the hard scattering of the two partons, \textit{a} and \textit{b}, to an intermediate parton (\textit{c}).
\item   $ D_{c \rightarrow h/jet}(z,Q^{2})$ is the fragmentation function (FF) that describes the probability that an outgoing parton, \textit{c}, fragments and hardonizes into a final state hadron, \textit{h}, within a jet with momentum fraction, $z \equiv p_{hadron} / p_{parton}$.
\end{itemize}

\afterpage{%
\begin{figure}[h]
\includegraphics[width=\linewidth]{ppcollison}
\centering
\caption{Schematic of a proton-proton collision.  Starting from the bottom, two partons confined within the colliding protons have a hard interaction.  The outgoing partons will induce partonic showers by radiating quarks and gluons.  The partonic showers will eventually form into final state hadrons, due to confinement, which are measured in high energy experiments\cite{Dobbs:2001ck}.}
\label{fig:FactorizationCartoon}
\end{figure}
\clearpage
}

\noindent
Figure \ref{fig:FactorizationCartoon} shows a cartoon of a pp collision broken into the relevant steps in accordance with the factorization theorem.  The best place to test QCD phenomena using hard probes, i.e. jets, is at high energy hadron colliders, such as those found at CERN\footnote{Discussed in detail in Chapter 3}, Fermilab, and BNL. The time scale that a hard probe is created in a high energy collision is on the order of $\tau \approx 1/p_{T} \approx$ \, 0.1 fm/\textit{c} at $p_{T} = 1 \,$ GeV, which corresponds to some of the earliest stages of the nuclear collision.  The factorization theorem is an incredible tool for understanding high energy interactions and the following sections will discuss the important concepts enveloped in it.

\subsubsection{Parton Distribution Functions}
The PDF occurs twice in Equation \ref{eq:xsection} due to the two partons that will undergo the hard scattering being confined in two different protons.  PDFs convey the structure of a nucleon in terms of the number of flavored quarks or gluons ($u(x)$, $d(x)$, $s(x)$, $\overline{u}(x)$, $\overline{d}(x)$, $\overline{s}(x)$, $g(x)$) and must obey certain constraints and summation rules.  In the case of a proton, with electric charge (\textit{e} = +1),

\begin{equation}
+1 = \frac{2}{3} \int_{0}^{1} [u(x) - \overline{u}(x)] dx - \frac{1}{3} \int^{1}_{0} [d(x) - \overline{d}(x)] dx
\label{eq:PDFcharge}
\end{equation}

\noindent
and isospin (\textit{I} = 1/2),

\begin{equation}
\frac{1}{2} = \frac{1}{2} \int_{0}^{1} [u(x) - \overline{u}(x)] dx - \frac{1}{2} \int^{1}_{0} [d(x) - \overline{d}(x)] dx
\label{eq:PDFIso}
\end{equation}

\noindent
have a solution,
\begin{equation}
 \int_{0}^{1} [u(x) - \overline{u}(x)] = 2
\label{eq:PDFSouU}
\end{equation}

\begin{equation}
\int^{1}_{0} [d(x) - \overline{d}(x)] dx = 1
\label{eq:PDFSouD}
\end{equation}

\noindent
This corresponds to the classical partonic view that protons contain two up quarks and a down quark similarly, and similarly the neutron, with charge \textit{e} = 0 and isospin I = -1/2, is composed of two down quarks and an up quark.  Naively, we could assume that the three quarks composing a proton would each carry a momentum fraction of approximately 1/3 the total momentum of a proton.  However, high energy deep inelastic scattering experiments conducted at the Stanford Linear Collider in the 1960's\cite{Panofsky:871460} measured the momentum carried by the three quarks as only accounting for about 1/2 the total proton momentum.  This led to a more complex and dynamic model of the proton structure with the other half of the proton momentum being carried by neutral partons, which would eventually become known as gluons.

\begin{figure}[h]
\includegraphics[width=12.0cm]{aOzz6}
\centering
\caption{Proton PDF at $Q^{2}$ = 10 GeV (left) and  $Q^{2}$ = 10 TeV (right) from the NNPDF Collaboration\cite{Feltesse:2010}.}
\label{fig:PDFNNPDF}
\end{figure}

Determining the structure of the partons making up a nucleon is a major endeavor by both theorists and experimentalists.  Two of the most popular PDFs available to physicists are the CTEQ\cite{Kovarik:2013sya} (Coordinated Theoretical-Experimental Project on QCD) and the NNPDF\cite{Ball:1966481} (Neural Network Parton Distribtuion Function) sets.  Figure \ref{fig:PDFNNPDF} shows the proton PDF as a function of the momentum fraction for two energy ranges.  At high values of \textit{x}, the two up quarks account for about 2/3 of the momentum fraction while the down quark accounts for about 1/3 of the total momentum.  These quarks are collectively called the valence quarks.  At high energies (low values of \textit{x}) we see that the proton has non negligible contributions from gluons, anti-quarks, strange, and even charm quarks.  These are collectively known as the sea partons.  Today, the modern picture of a proton's structure is mostly composed of gluons and sea quarks at low values of \textit{x} and this domination only increases as a function of $Q^{2}$\cite{Fritzsch:1992mu}.

\subsubsection{Parton-Parton Cross-Section}
The quark-pquark, quark-gluon, and gluon-gluon cross section can be calculated using perturbation theory.  To the zeroth order in $\alpha_{s}$ this cross-section would be a simple quark-antiquark annihilation and would be calculable using Feynman diagrams as seen in Figure \ref{fig:qqbar}\cite{Collins:1989gx}.  Higher ordered contributions, such as the creation of virtual gluons, require the hard cross-section to be expanded as a series in terms of $\alpha_{s}$.  Calculations of the hard cross-section that incorporate these higher order terms are known as \textit{next-to-leading order} (NLO) with N denoting the number of terms after the leading order that have been included in the cross-section calculation.  Various calculations of the hard cross-section of different QCD processes have been performed over the years typically using either power series or logarithmic expansions of $\alpha_{s}$\cite{Brambilla:2006wp} and corrections for LO, NLO, and even NNLO constitutes a very active field in high energy physics.  Perturbative techniques of the hard cross-section have been extremely successfully in describing jet features in hadronic collisions\cite{Fritzsch:1992mu}.

\begin{figure}[h]
\includegraphics[width=6.0cm]{Ttbar_production_via_qqbar_annihilation}
\centering
\caption{Lowest order quark-antiquark annihilation to top-antitop pair\cite{Erdmann:2001ne}.}
\label{fig:qqbar}
\end{figure}

\subsubsection{Hadronization}

Hadronization, the process by which the colored pQCD partons form into colorless non-pQCD hadrons, represents a significant barrier in progressing jet physics.  This is due to the fact that hadronization encompasses several smaller processes, which in themselves are hard to characterize. Thus, like PDFs, an accurate description of hadronization requires a phenomenological approach by which experimental results help complement theoretical calculations.  Jet production via hadronization\cite{Webber:1994zd} follows two distinct stages.  First, the partons that underwent a hard scattering start to emit radiation via gluon bremsstrahlung up until time, $t < Q^{2}$.  This is known as the parton cascade.  The parton cascade is the precursor to a jet as most of the radiation generated will travel in the same direction as the initial hard scattered parton.  However, this immediately poses an issue in jet physics as radiation generated at a wide angle away from the momentum axis of the initial hard scattered parton will not be associated with the jet.

\begin{figure}[h]
\includegraphics[width=8.0cm]{partoncascade}
\centering
\caption{Parton cascade in a hadronic collision\cite{Webber:1994zd}.}
\label{fig:pcascade}
\end{figure}

\noindent
After the cascade has ended, the partons form into color neutral hadrons.  There are two main phenomenological models used to describe the hadron forming process, the Lund String Model and the Cluster Hadronization Model.  

The QCD potential is

\begin{equation}
V(r) = - \frac{\alpha_{s}}{r} + \sigma \, r
\label{eq:QCDPotential}
\end{equation}

\noindent
where the first term of Equation \ref{eq:QCDPotential} is similar to the Coulomb potential with a 1/r dependence and is the dominate term at short distance.  The second term has a string-like potential with $\sigma$ referring to a string-like tension.  The Lund String Model uses this potential, ignores gluon radiation, and has fragmentation occur via breaking the string tension with the production of $q\overline{q}$ sea quarks.  The created sea quarks will carry some momentum fraction, z, of the initial parton until z falls below some cutoff.  Figure \ref{fig:qqbarstring} shows two quarks undergoing a string breaking, each of the quarks initiating the string breaking will combine with a sea quark in an iterative manner to form hadrons.  


\begin{figure}[h]
\includegraphics[width=15.0cm]{qqbarproductionvaccum}
\centering
\caption{$u \overline{d}$ generating a $d \overline{d}$ pair via string breaking which will form color neutral hadrons, black lines show the string like equipotentials.\cite{Andersson:2002ap}.}
\label{fig:qqbarstring}
\end{figure}

The Cluster Hadronization Model has gluons splitting after the parton cascade phase into $q\overline{q}$ pairs.  These pairs will form color-singlet clusters with other neighboring quarks in phase-space.  These color-singlets will typically be a few GeV/\textit{$c^{2}$} in mass and are treated as excited meson resonances.  These psuedo-resonances will decay via their normal branching ratios into the stable hadrons\cite{Webber:1983if}.

\subsubsection{Fragmentation}

Similar to the way a PDF quantitatively describes the structure of a nucleon, the fragmentaion function (FF) quantitatively describes the hadronization process.  The FF is also similar to the PDF in that it is also a probability distribution, thus it follows the probabilistic rule that

\begin{equation}
\sum \int_{0}^{1} z D_{c \rightarrow \, h/jet} (z,Q^{2})dz = 1
\label{eq:FFRule}
\end{equation}

\noindent
where the sum is carried over the particles constituting the jet, $c \rightarrow h/jet$ states that the function in question is only concerned with a parton, c, fragmenting into a final state particle, h, that is part of a jet.  The fractional momentum of the hadrons created from the fragmenting parton, $z \equiv p_{hadron} / p_{parton}$, is an exponentially decreasing distribution between 0 and 1 which shows how fragmented hadrons carry the partial energy from the initial parton scattering.  Parton-Hadron Duality\cite{Jenkovszky:2012dc} states that the leading hadron should correlate with the kinematic properties associated with the hard scattered quark that initiated the jet.  Thus we can measure the fragmentation function as $z = p_{hadron} / p_{jet}$.  The formulation of the FF as the fractional energy carried by the hadrons in a jet was a breakthrough in pQCD techniques and is analogous to the way an electron passing through an absorber creates photon showers. These photons continue generating conversion electrons until the total energy has been dissipated into the material.

\begin{figure}[h]
\includegraphics[width=8.0cm]{FFfunctions}
\centering
\caption{Fragmentaion functions from $e^{+}e^{-}$and DIS experiments with fits\cite{rak_tannenbaum_2013} as a function of the total cross-section, $\sigma$.}
\label{fig:FFfunc}
\end{figure}


Figure \ref{fig:FFfunc} is the FF in terms of the Gaussian equation, with $\sigma$ refering to the total cross-section, $z \, dN/dz = - dN /d \xi $, and $\xi = -ln  \,1/z_{p}$. The Gaussian peaks in Figure \ref{fig:FFfunc} along with the suppression of the FF at low z values due to gluon coherence were predicted by pQCD. 

\section{Jet Finding Algorithms}

A jet arises from the fragmentation of a hard parton to final state hadrons.  However, grouping the hadrons together into a jet is ambiguous.  Jet finding algorithms are used because they standardized the definition between theorists and experimentalists and give results comparable to each other. Early on in jet physics, both theorists and experimentalists used a wide variety of jet finders and definitions which made comparisons between experiments or to theoretical calculations nearly impossible\cite{Atkin:2015msa}.  For example, a radiated gluon that splits into a quark anti-quark pair may become one or two jets depending on the angular separation and the algorithm used.  Early jet finders tended to be sensitive to soft particles or could give widely varying yields to the number of jets in an event.  In 1990, the Snowmass Accord\cite{Huth:217490} reached a standardized definition of a jet between experimentalists and theorists.  The agreement maintained that any algorithm that clusters particles into a jet must be both infrared and collinear safe (IRC).  

\begin{figure}[h]
\includegraphics[width=15.0cm]{IRCsafe}
\centering
\caption{Cartoon showing collinear and infrared safe jet candidates\cite{Blazey:2000qt}.}
\label{fig:IRCsafe}
\end{figure}

A hard, high momentum transfer, scattered parton will undergo collinear splittings, emissons of gluons, as part of the fragmentation process.  This is a difficult process to model theoretically so jet finding algorithms maintain collinear safety.
Collinear safety ensures jets remained unchanged due to any gluon emissions.  Infrared safety in turn requires that the emission of soft radiation should not affect jet.  This makes jets returned by the algorithm  a signature of a hard process.  Both of these processes are shown in Figure \ref{fig:IRCsafe}. After the adoption of these standards from the Snowmass Accord, old algorithms that violated these rules were patched and new jet finders were developed to comply with IRC safety.  The most prevalent jet finding algorithms today fall into two categories: cone algorithms and sequential recombination/clustering algorithms.

\subsubsection{Cone Algorithms}

Cone algorithms made up the bulk of early jet finders.  The only IRC safe cone algorithm still in use today is the seedless infra-red safe cone algorithm (SIScone).  SIScone defines a cone of radius R around the highest momentum particle in the coordinates of $(\eta,\phi)$\footnote{It is possible to use a Cartesian coordinate system in particle colliders, with the z-component referring to points along the beam axis while the xy-plane is perpendicular to the beam axis.  However, this system is not invariant under a Lorentz boost.  Therefore it is more useful to use the cylindrical-like coordinates of psuedorapidity ($\eta$) and the azimuth angle ($\phi$). Psuedorapidity may be thought of as the polar angle in a cylindrical coordinate system with $\eta = 0$ when the polar angle is perpendicular to the beam axis and $\eta = \infty$ along the beam axis.  $\phi$ is the azimuth angle that rotates around the beam axis.  Both, $\eta$ and $\phi$ are invariant for Lorentz boosts along the beamline and allow for easy comparisons between the center-of-mass frame and the laboratory frame of a high energy collision.}.  This is the proto-jet.  SIScone then proceeds through an iterative process of finding all the particles within the jet radius such that $R \leq \sqrt{\phi^{2} + \eta^{2}}$ and calculates a new jet center based on these particles' momenta and a new weighted jet axis$(\eta,\phi)$.  If the new center matches the proto-jet center, the proto-jet is tagged as a stable jet.  All the particles in that jet are removed and SIScone moves onto the next highest $p_{T}$ particle.  Cone algorithms tend to be unpopular due to being computationally expensive, difficult to implement theoretically, and can give results not calculable in perturbation theory.

\subsubsection{Sequential/Recombination Algorithms}

The other class of jet finders are the sequential/recombination algorithms, which are favored by experimentalists and theorists, and are IRC safe.  There are three sequential/recombination algorithms: $k_{T}$, Anti-$k_{T}$, and the Cambridge/Aachen jet finders, with $k_{T}$ referring to the component of a jet constituent's momentum perpendicular to the jet axis.  All of the algorithms use a similar method.  First they find the distance between every pair of particles, $d_{i,j}$,  such that



\begin{equation}
d_{i,j} = min[p^{a}_{T,i},p^{a}_{T,j}] \, \frac{\Delta^{2}_{ij}}{R^{2}}
\label{eq:JetAlgo}
\end{equation}

\noindent
where $p^{a}_{T,i}$ is the transverse momentum of particle \textit{i}, \textit{a} is free parameter that is set based on which algorithm is used, $\Delta^{2}_{ij} = (\eta_{i} - \eta_{j})^{2} + (\phi_{i} + \phi_{j})^{2}$ is the distance between the particles, and R is the radius of the jet.  A second distance is defined in the sequential/recombination algorithm scheme,

\begin{equation}
d_{i,B} = p^{a}_{T,i}
\label{eq:MinJet}
\end{equation}

\noindent
which is only a function of the particles transverse momentum.  Sequential/Recombination algorithms find the set of all particles, ${d_{i,j},d_{i,B}}$, such that if $d_{i,B}$ is the minimum for particle \textit{i} it is tagged as a jet and removed from the list.  If $d_{i,j}$ are a minimum for particles \textit{i} and \textit{j} these two particles are merged together into a new particle (\textit{ij}) and a new minimum is found between (\textit{ij}) and a new particle \textit{k} until all the particles are either merged into jets or the minimization function is no longer satisfied.

\subsubsection{$k_{T}$ Algorithm}
The $k_{T}$ algorithm sets the value \textit{a} to 2, this results in a minimization function,

\begin{equation}
d_{i,j} = min[p^{2}_{T,i},p^{2}_{T,j}] \, \frac{\Delta^{2}_{ij}}{R^{2}}
\label{eq:kt}
\end{equation}

\noindent
which clusters low momentum particles first, making this algorithm susceptible to the underlying event, UE, or pile-up, PU.  Thus the $k_{T}$ algorithm is good at estimating any background present in a high energy collision. 

\subsubsection{Anti-$k_{T}$ Algorithm}
The Anti-$k_{T}$ algorithm sets the value \textit{a} to -2, resulting in a minimization function,

\begin{equation}
d_{i,j} = min \Bigg [\frac{1}{p^{2}_{T,i}}, \frac{1}{p^{2}_{T,j}} \Bigg ] \, \frac{\Delta^{2}_{ij}}{R^{2}}.
\label{eq:Akt}
\end{equation}

The minimization function begins with high-$p_{T}$ particles, thus the area and axis of a jet is only slightly perturbed by soft particles.  This makes the Anti-$k_{T}$ algorithm robust in jet finding with events having pile-up.  The Anti-$k_{T}$ algorithm is the default jet finding algorithm used at the Large Hadron Collider and is the one used in this thesis.

\subsubsection{Cambridge/Aachen Algorithm}

The Cambridge/Aachen algorithm sets \textit{a} to 0 and this results in a minimization function of,

\begin{equation}
d_{i,j} = \frac{\Delta^{2}_{ij}}{R^{2}}
\label{eq:CBalg}
\end{equation}

\noindent
which makes it independent of particle momentum and sensitive to PU and the UE.  Due to the fact that the Cambridge/Aachen algorithm is only dependent on the particle coordinate it is most useful in studying jet structure.

Figure \ref{fig:AllJetFinder} shows the jets found in a single event using all four jet finding algorithms.  It should be noted that the Cambridge/Aachen and $k_{T}$ algorithms have highly irregular and large shapes, making them both susceptible to the presence of a UE, while SIScone finds an additional jet due to splitting.  The Anti-$k_{T}$ algorithm finds circular jets which demonstrates its' robustness to hard radiation.  

\afterpage{%
\begin{figure}[h]
\includegraphics[width=\linewidth]{FastJet}
\centering
\caption{Lego plot of all four jet finders used on a single event with R = 1 jet radius\cite{Atkin:2015msa}.}
\label{fig:AllJetFinder}
\end{figure}
\clearpage
}

Once a stable jet is found, a recombination scheme is deployed in order to garner the jet kinematics.  By adding the 4-vector, $\boldsymbol{p}^{\mu} = (\boldsymbol{E},\boldsymbol{p}_{x},\boldsymbol{p}_{y},\boldsymbol{p}_{Z})$, for all of the associated particles composing a jet, we may obtain the jet momentum, energy, coordinates, etc.  In a particle collider with the tracks from a tracking detector measuring particle momentum and the towers of a calorimeter measuring particle energy we obtain the following relationships



\begin{equation}
p_{T}^{jet} = \sum_{particles} p_{T} = \sum_{tracks} p_{T}
\label{eq:JePt}
\end{equation}

\begin{equation}
E^{jet} = \sum_{particles} E = \sum_{towers} E
\label{eq:JetE}
\end{equation}

\begin{equation}
\eta^{jet} = \frac{1}{2} \, \ln \Bigg (  \frac{|\boldsymbol{p}^{jet}| + p_{L}^{jet}}{|\boldsymbol{p}^{jet}| - p_{L}^{jet}}  \Bigg )
\label{eq:JetEta}
\end{equation}

\begin{equation}
\tan \phi^{jet} = \frac{p_{y}^{jet}}{p_{x}^{jet}}
\label{eq:JetPhi}
\end{equation}
\noindent 
where $p_{L}$ refers to the longitudinal momentum which is the momentum component parallel to the beam axis, $\eta^{jet}$ and $\phi^{jet}$ are the jet coordinates in psuedo-rapidity and the azimuth angle.  This method of adding the 4-vector of the particles composing the jet together in order to gain the jet kinematics is known as the E-scheme\cite{Cacciari:2011ma}.

\subsubsection{FastJet}
FastJet\cite{Cacciari:2011ma} is a \verb|C++| software package that performs jet finding.  Due to the computational efficiency, ease of use, and straight forward implementation, FastJet is the preferred jet finding software package used by theorists and current high energy experiments. It implements the four previously discussed jet finders along with both the E-scheme and a boost invariant $p_{T}$ scheme (BIpt-scheme) for recombination.  The BIpt-scheme obtains the jet momentum and energy in the same manner as the E-scheme but uses a weighted average to find the jet coordinates,

\begin{equation}
\eta^{Jet} = \sum_{particle} \frac{p_{T}^{particle}}{P_{T}^{jet}} \, \eta^{particle}
\label{eq:JetEtaRecom}
\end{equation}
\begin{equation}
\phi^{jet} = \sum_{particle} \frac{p_{T}^{particle}}{P_{T}^{jet}} \, \phi^{particle}
\label{eq:JetPhiRecom}
\end{equation}

\noindent
In addition to basic jet measurements, FastJet contains a number of advance features, which allows it to be used to study jet area, jet substructure, and jet background subtraction\cite{Connors:2017ptx}.

\section{Monte-Carlo Generators}
Monte Carlos allow for the simulation of high energy events on a statistical basis.  Particle level generators use different phenomenological models of the factorization theorem in order to simulate the energy, momentum, particle species, multiplicity, and direction of travel expected in a high energy collision.  In this thesis Monte Carlos are also used to understand and correct for inefficiencies due to the experiment using a GEANT simulation of ALICE, this is discussed in more depth in Chapter 5.  The following sections will go over some of the different Monte Carlos used in this thesis and the physics behind how they simulate high energy collisions.

\subsubsection{PYTHIA}

PYTHIA\cite{Sjostrand:2007gs}, is a Monte Carlo software tool-kit used to model proton-proton collisions.  The package uses pre-defined parton distribution functions as input.  Afterwards it simulates the partonic showers and radiation due to a hard scattering by generating the leading-order, LO, scattering matrix elements.  Hadronization is performed in PYTHIA using the Lund String Model.  The final state hadrons are formed using the branching ratios to decay excited states.

PYTHIA underestimates jet production due to the limitations of using LO calculations.  Therefore, it uses an arbitrary value (K-factor) to make NLO corrections to the LO cross section.  The K-factor is defined as

\begin{equation}
K = \frac{\sigma_{NLO}}{\sigma_{LO}}.
\label{eq:Kfactor}
\end{equation}

NLO corrections to the cross-section will not match experimental results, especially at low energies.  PYTHIA implements additional phenomenological adjustments used to better match data.  PYTHIA encompasses these parameters into sets known as `tunes', with PYTHIA 6.4 Perugia-2010 tune being used for this analysis\cite{Skands:2010ak}.

\subsubsection{PHOJET}
PHOJET is a \verb|FORTRAN 77| Monte Carlo simulator used to model proton-proton collisions. It is an alternative to PYTHIA and is better at modeling soft physics processes present in high energy collisions.   PHOJET implements the Dual Parton model\cite{CAPELLA1994225}\cite{Wong:241251} and multiple parton interactions\cite{Bopp:1998rc} to model soft physics, similar to PYTHIA.  Hard interactions are implemented in PHOJET using LO scattering elements and it uses PYTHIA for the fragmentation and hadronization phase.  Due to its ability to model soft physics, PHOJET is better at comparing to Min Bias\footnote{Events with a low total transverse momentum and high cross section} data and understanding jet results in a low kinematic range.  PHOJET also acts as a benchmark in understanding any bias due to using other Monte Carlo generators, such as PYTHIA.  PHOJET v1.2 is used in this thesis.


\subsubsection{HERWIG}
The HERWIG\cite{Bahr:2008pv} Monte Carlo generator is a \verb|FORTRAN| software package used to generate proton-proton events.  It is similar to PYTHIA in that it calculates the LO hard scattering of partons, however it uses the cluster model of hadronization to produce jets based on gluon splitting.  It is also similar to PHOJET in regards to the evolution of final state jets with soft gluon angular ordering.  By comparing HERWIG, PYTHIA, and PHOJET it is possible to test for sensitivities to jet production in high energy events due to different types of hadronization models and soft radiation.


\section{The Quark-Gluon Plasma}
At the temperatures and pressures typical to the universe today nuclear matter is confined to a colorless hadrons.  However, it was theorized that at extreme temperatures, such as those experienced in the early universe, partons would have undergone a phase transition where they were no longer bound in a color neutral state.  This state of matter would have been analogous to a conventional plasma where the electrons are no longer bound to a nucleus, thus the state was dubbed the Quark-Gluon Plasma (QGP).

The nuclear phase diagram is shown in Figure \ref{fig:QCDphase} as a function of temperature and the net baryon density.  Normal nuclear matter is confined to the bottom left while increasing temperatures and/or densities correspond to the QGP.  Modern particle colliders, such as RHIC and the LHC, are able to obtain the densities and temperatures necessary to create a QGP and are likewise shown in the figure.  The reason for particle colliders being located at low baryon density is due to the fact that at collider energies at mid-rapidity the plasma are dominated by quark-antiquark pairs, so the net baryon density is close to zero.  This dilutes the total baryon density in the initial system and is more akin to what the early Universe was like.  

\begin{figure}[h]
\includegraphics[width=8.0cm]{610_1}
\centering
\caption{The QCD phase diagram\cite{Mohanty:2013yca}.}
\label{fig:QCDphase}
\end{figure}

\subsubsection{Nuclear Collisions}
By colliding heavy nuclei together in high energy colliders it is possible to obtain the energy densities and temperatures associated with the QGP.  The first signatures for the QGP were measured via a J/$\psi$ suppression at the Super Proton Synchrotron, located at CERN in 2000\cite{Csorgo:2000yu}.  In 2005, the four experiments on the RHIC collider: BRAHMS\cite{Arsene:2004fa}, PHENIX\cite{Adcox2005184}, PHOBOS\cite{Back200528}, and STAR\cite{Adams2005102}, co-announced the observation of a new state of matter consistent with the hot and dense QGP.  The results from RHIC indicated that the QGP behaves more like a perfect fluid over a plasma-like state\cite{Jacak310}.

Figure \ref{fig:HeavyIonCollisionvPP} shows the difference between a proton-proton collision and a heavy-ion collision.  The heavy-ion collision mirrors the processes in a proton-proton collision (left) described in depth in Section \ref{sec:fac}.  After the initial hard scattering the phase transition to a QGP occurs.  The QGP undergoes a hydrodynamical evolution and expansion until it cools to a colorless hadronic gas.  After the phase transition occurs  hadrons will undergo chemical reactions until the final particle species is set, once these reactions cease we have a chemical freeze-out. The hadron gas continues to expand and cool until all soft elastic interactions and momentum transfers cease.  This is the kinetic freeze-out, after which the final momentum spectra is set.  Understanding how the final particle composition accounts for the measured light-nuclei seen in heavy-ion collisions was the topic of a paper I published and more information about this subject in heavy-ion physics can be found here\cite{Sharma:2018dyb}.

\afterpage{%
\begin{figure}
\includegraphics[width=11.0cm]{ppevolution}
\centering
\caption{Comparison of the processes in a proton-proton collision with no medium and a heavy-ion collision with a colored medium stage\cite{PhysRevD.27.140}.}
\label{fig:HeavyIonCollisionvPP}
\end{figure}
\clearpage
}

\subsubsection{Jets and The QGP}

Jets are an excellent probe of the properties of the QGP.  Jets are produced in the earliest stages, before the formation of the QGP, and survive the full evolution of a heavy-ion collision.  As a jet propagates through the QGP, it will lose energy to the medium through a combination of gluon radiation to the colored medium and inelastic scatterings.  These energy loss mechanisms are dependent on the distance a parton travels through the QGP and on the species of the parton.  


\begin{figure}[h]
\includegraphics[width=4.5cm]{dijetfig2}
\centering
\caption{Jet energy loss in a QCD medium\cite{Mohanty:2013yca}.}
\label{fig:JetEloss}
\end{figure}

Figure \ref{fig:JetEloss} shows two back-to-back partons undergoing a hard scattering.  Both will fragment into jets, but the first parton with transverse energy, $E_{T1}$, will be subjected to much less energy loss over the second parton due to the first parton only traveling through the outer edge of the QGP.  The species dependent partonic energy loss arises from kinematic constraints to gluon emission from the heaviest of quarks.  This radiation is suppressed at angles smaller than the ratio of the quark mass to its energy and has been dubbed the \textit{Dead-Cone Effect}\cite{Thomas:2004ie}.  Tagging the flavor dependence of jets, either via measuring electrons from semi-leptonic decays or reconstructing the secondary vertex of heavy flavor mesons, has recently shown that energy loss via the Dead-Cone Effect is strongly suppressed with jets containing a charm quark\cite{CAO2018255}.

One way of quantifying the energy loss in a heavy-ion collision is via measurements of the nuclear modification factor, $R_{AA}$,


\begin{equation}
R_{AA} = \frac{1}{N_{binary}} \frac{d^{2}N_{AA}/dp_{T}d\eta}{d^{2}N_{pp}/dp_{T}d\eta}
\label{eq:RAA}
\end{equation}

\noindent
where $N_{binary}$ is the number of nucleon-nucleon collisions and is estimated using a Glauber model\cite{Miller:2007ri} of a nucleus while $d^{2}N_{AA}/dp_{T}d\eta$ and $d^{2}N_{pp}/dp_{T}d\eta$ are the spectra measured in nucleus-nucleus and proton-proton collisions respectively.  $R_{AA}$ may be thought of as asking the question: Does a heavy-ion collision scale as a superposition of $N_{binary}$ nucleon-nucleon collisions?  A $R_{AA}$ value of 1 corresponds to no modification in a heavy ion collision not already present in a proton-proton collision.  The observation of $R_{AA}$ below unity shows a suppression of jets in heavy-ion collisions.  Where does the missing energy go?  This is still a subject for debate and it is not clear whether the energy may propagate outside of the cone radius of the jet or if the energy may become thermalized in the medium.


Figure \ref{fig:JetRAA} shows the nuclear modification factor with R = 0.4 jets in the ATLAS experiment at 5.02 TeV\cite{Aaboud:2018twu}.  The different colored bands in the figure are centralities\footnote{The purple 10 - 20\% band denotes the most central events (i.e. the two colliding nuclei have a low impact parameter and collide nearly head-on), while the 70 - 80\% red band denotes the least central events (i.e. the two colliding nuclei have a high impact parameter and barely graze one another).  An in depth discussion of centrality may be found here\cite{Klochkov_2017}.}

\begin{figure}[h]
\includegraphics[width=10.0cm]{jetRAA}
\centering
\caption{Jet $R_{AA}$ at 5.02 TeV with the ATLAS experiment\cite{Aaboud:2018twu}.}
\label{fig:JetRAA}
\end{figure}



\subsubsection{Collectivity in Proton-Proton Collisions}
As previously stated a QGP is believed to be absent in proton-proton collisions, thus any signature of a QGP should likewise be absent.  However, one way of quantifying the presence of the QGP is via the Bjorken energy density.  

\begin{equation}
\varepsilon = \frac{1}{\tau A} \frac{dE_{T}}{d \eta}
\label{eq:bjorkenEt}
\end{equation}

\noindent
where A is the transverse area of the nuclei, $\tau$ is the proper time, and $dE_{T}/d \eta$ is the transverse energy per unit psuedorapidity.  It can be shown that  the 150 MeV critical temperature need for the phase transition to the QGP corresponds to ~ 1 - 3 GeV/$fm^{3}$ energy density.  The quantity $dE_{T}/d \eta$ can be related to the mean transverse momentum $<p_{T}>$ and particle multiplictiy\footnote{Multiplicity is defined as the number of particles per event} per unity psuedorapidity as:

\begin{equation}
\frac{dE_{t}}{d \eta}  \approx  <p_{T}> \frac{dN}{d\eta}
\label{eq:Et}
\end{equation}

where $ <p_{T} >$ is the mean transverse momentum and $dN/d\eta$ is the particle multiplicity per unit psuedorapidity.  This suggests that in very high multiplicity proton-proton events signatures of the QGP may be present.  Although suppression has never been observed in high multiplicity proton-proton collisions, physicists have recently measured azimuthal correlations in such systems\cite{Nagle:2018nvi}.  This gives a `hint' that flow may be present in high multiplicity proton collisions.  CMS presented results in proton-proton collisions at 13 TeV using soft-particles, $p_{T} \leq\,$ 2 GeV/\textit{c}, consistent with hydrodynamical predictions\cite{ZHAO2018495}. These results have opened new debates and questions into the very nature of the QGP.  Measuring jets to high accuracy over a wide kinematic range is important because it serves as a baseline measurement for the inducing the QGP properties in heavy-ion collisions.  This will be explored in more detail throughout the rest of this thesis.
\chapter{Introduction} \label{ch:introduction}

From the Vedas to the ancient Greeks, generations have described the constituents of nature in terms of indivisible `elements'.  It wasn't until the beginning of the 20th century that the ancient elements of earth, wind, fire, water, and the aether were abandoned for the atomic theory of nature.  By the 1960's, what would become known as the Standard Model of Particle Physics was taking shape.  The five ancient elements were replaced by the fundamental particles: quarks and leptons comprising the spin 1/2 fermions and the force carrying bosons with spin 1 as seen in Fig. \ref{fig:fundpart}
\begin{figure}[h]
\includegraphics[width=6.0cm]{Fundamental_Particles}
\centering
\caption{The fundamental particles of the Standard Model\cite{Patrignani:2016xqp}.}
\label{fig:fundpart}
\end{figure}
\par
The Standard Model is the unification of the three symmetry groups, SU(3) x SU(2) x U(1), representing the strong, weak, and electromagnetic forces\cite{Langacker:2009my}.  In terms of scientific accomplishments the Standard Model is one of the most tested theories of nature with an agreement between the theory and observed results up to ten digits\cite{Aoyama:2014sxa}. 
 
 \noindent
Even though the Standard Model gives us a deep understanding for many natural phenomena and has a wide range of uses; understanding the evolution of the Big Bang, how atoms and molecules bond, the nature of light, to cancer treatments and nuclear security it is fundamentally an incomplete theory of nature.  The fact that Gravity has yet to be unified into a quantum theory tells us that the Standard Model is incomplete.  High energy experiments give us some of the most extreme conditions possible to test the Standard Model and to look for phenomena outside of the theory.  Are their new symmetries and laws that manifest at high energies? Can we create dark matter or dark energy in a laboratory?  Are quarks and leptons fundamental or finite in size?  Do the four fundamental forces emerge from some yet unknown unified force?  And why is antimatter absent in the Universe?  All of these open questions are of great interest and currently form large areas of active research.  

\par
The theory of strong interactions, Quantum Chromodynamics(QCD), is described by the SU(3) group and similar in analogy to the electric charge of Quantum Electrodynamics (QED) carries color charge, red, green, and blue.  Quarks and gluons are colloqually known as partons and are particles that interact via the strong force.  At low energies and over large length scales partons are confined to a color neutral state and these particles must clump together into color neutral hadrons.  As two colored partons began to separate, at some point it becomes energetically favorable to create a quark--antiquark pair out of the vacuum rather then expanding the distance between neighboring partons.  Due to confinement, quark interactions at high energy collider experiments manifest themselves as a spray of hadrons known as a 'jet'.  The other main attribute of QCD is asymptotic freedom, as the interactions between partons becomes more energetic and the legth scale decreases  the strong coupling constant becomes exceedingly small, $ \alpha_{strong} << 1$, and the partons freely interact with oneanother.  Due to asymptotic freedom nuclear matter undergoes a phase transition called the Quark--Gluon Plasma (QGP) at high energies and densities 



%The theory of strong interactions, Quantum Chromodynamics(QCD), is described by the SU(3) group and similar in analogy to the electric charge of Quantum Electrodynamics carries color charge, red green, blue.

%Nuclear collisions, were the strong coupling constant becomes vanishingly small, allows for 
%\par 
%The Large Hadron Collider(LHC) is the largest and most powerful particle collider ever built.  In 2012 it gained world wide attention for the discovery of the Higgs Boson made by the Compact Muon Solenoid(CMS) and A Toroidal LHC ApparatuS(ATLAS) experiments.  By colliding lead atoms with one another the LHC is able to achieve the temperatures and energy densities necessary to create the QGP.  However, at the LHC the QGP only has a mean lifetime of $ ~ 10^{-23}$  seconds and thus measuring the properties of this phase of matter is difficult and must be infered by the final state particles that reach the detectors. Jets are an attractive tool for measuring the properties of the QGP because they created during the earliest stages of a heavy ion collision


\par
This thesis will present an overview of Quantum Chromodynamics in Chapter 2, with an emphasis on jet physics and heavy ion collisions.  Chapter 3 will give a brief overview of the Large Hadron Collider and the ALICE experiment, including the relevant subsystems for this jet analysis.  Jet results from the ALICE detector along with comparissons to QCD simulations with systematic uncertainty calculations and unfolding for detector effects will be given in Chapter 4.  The ALICE time projection chamber will be upgraded to a continous readout mode, the physics motivation behind this and author's contirbutions will be discussed in Chapter 5.  Chapter 6 will serve as a final discussion, outlook, and conclusion. 
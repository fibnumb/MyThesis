\chapter*{Abstract}\label{ch:abstract}

Hadronic jets are created in the earliest stages of a high energy collision due to a hard interaction between two partons.  ALICE (A Large Ion Collider Experiment) is one of the main detectors located at the Large Hadron Collider.  By using the ALICE Time Projection Chamber and Electromagnetic Calorimeter inclusive jets can be reconstructed over a wide kinematic range.  Measuring jets in proton-proton collisions serves as a baseline measurement for understanding energy loss in the nuclear matter and the processes by which quarks form into hadrons.  

Inclusive jet cross-sections and their ratios were measured in this thesis using the 8 TeV ALICE proton-proton data.  Results are shown for jets of radii 0.2, 0.3, and 0.4 and between 20 GeV/\textit{c} and 100 GeV/\textit{c}.  This thesis combines data from both the Minimum Bias and high-$p_{T}$ triggers.  Cross-sections were corrected for detector effects using a bin-by-bin approach and the results are compared to different Monte Carlo models and previous measurements. 
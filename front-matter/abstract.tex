\chapter*{Abstract}\label{ch:abstract}
In nuclear collisions a deconfined state of partons, quarks and gluons, interact creating a near \lq perfect\rq \, fluid called the Quark\textendash gluon plasma(QGP).  As this state of matter expands and cools, the quarks once again combine into the hadrons measured in particle detectors.  Understanding the energy loss mechanisms of this phase of matter is one of the major goals of the nuclear physics program at the Large Hadron Collider(LHC).  

The first principle processes that govern hadronization are not understood from a theoretical Quantum Chromodynamic (QCD) framework, but are well described phenomenologically using Monte Carlo simulations.  
As partons interact with one another, they fragment into collomated sprays of particles known as jets.  The topologies and properties of hadronic jets measured in collider experiments can be correlated to the hadrionzation phase in nuclear collision.   Measuring  jet cross-sections as a function of the jet radius in high energy experiments can constrain different hadronization models from one another and allow for more precise measurements of the QGP.

Inclusive jet cross sections and ratios of jet cross sections are measured in this thesis using the 2012 proton\textendash proton data collected at $ \sqrt{ \mathrm{s} } = $ 8 TeV with the ALICE detector at the LHC.  This thesis presents results of jets with radii from 0.2 to 0.4, over a wide kinematic range between 20 GeV/\textit{c} and 100 GeV/\textit{c} in momentum.  This thesis used both minimum bias and single shower triggered data from the ALICE Electromagnetic Calorimeter.  The cross sections are corrected for detector effects by unfolding and the results are compared to Monte Carlo simulations using Pythia, Herwig, and PHOJet in order to gauge different hadronization effects.